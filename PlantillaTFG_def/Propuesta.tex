\chapter{La proposta de treball de final de grau}\label{proposta}

\section{Principis b sics}
En un treball de final de grau ben planificat, la proposta de \ac{TFG} constituir  el primer document formal relatiu al projecte. A m s, especialment amb la introducci  del contracte docent relatiu al \ac{TFG}, la proposta assoleix una import ncia cabdal ja que aquest  s el document sobre el qual estudiant, director i \ac{EPS} adquireixen compromisos respecte de la duraci  m xima del treball, tasques de supervisi , disponibilitat de materials, \ldots

La redacci  d'una proposta ben definida i consensuada entre director i alumne suposa avantatges importants com, per exemple:
\begin{itemize}
\item Garanties de que, abans d'iniciar el desenvolupament del \ac{TFG}, l'alumne compr n perfectament el problema que es vol abordar i  s capa  de contextualitzar-lo, entenent com la tem tica del seu \ac{TFG} lliga amb la resta de coneixements de l' rea.

\item Disponibilitat d'una descripci  precisa dels objectius del \ac{TFG} i d'una estimaci  dels resultats que se n'esperen.

\item Visualitzaci  del full de ruta del \ac{TFG} que ha de permetre, tant al director com a l'alumne, l'avaluaci  del progr s en l'execuci  del projecte i en l'assoliment dels objectius.
\end{itemize}

En la major part dels casos, la proposta formal d'un \ac{TFG} es redactar  a partir d'una proposta de tema provinent d'un professor o d'un grup de recerca. Aquesta proposta de tema descriur  breument el context del treball, el problema adre at i els objectius que s'intenten assolir. A m s, hauria de proporcionar les refer ncies i eines b siques a utilitzar i explicitar, tamb , els coneixements previs i habilitats recomanades per poder dur a terme el \ac{TFG}. El m s desitjable seria que totes les propostes de temes de \ac{TFG} es fessin p bliques a trav s de la web de l'\ac{EPS}, tanmateix  s probable que tamb  n'apareguin als taulers d'anuncis, portes dels laboratoris de recerca o les portes dels despatxos dels professors.  s responsabilitat dels alumnes estar atents a la publicaci  d'aquests anuncis.

Es pot donar el cas d'alumnes que prefereixin realitzar el \ac{TFG} en una empresa o alumnes que estiguin especialment interessats en un  rea de coneixement concreta i ells mateixos vulguin proposar un tema de \ac{TFG}. En qualsevol d'aquests casos  s important que trobin un supervisor i que discuteixin amb ell la preparaci  de la proposta formal de \ac{TFG}. En el cas de \ac{TFG}s desenvolupats en empreses ser  important tamb  comptar amb el vist-i-plau del director/tutor dintre de l'empresa.

El m s habitual ser  que l'alumne interessat en una proposta de tema de \ac{TFG} comenci llegint les refer ncies b siques subministrades per tal d'assolir uns coneixements b sics sobre el context general del tema proposat i poder plantejar el problema que es vol abordar i els objectius concrets que es persegueixen. Aix , comptant amb el suport del supervisor del \ac{TFG}, ha de permetre passar a l'etapa de redacci  de la proposta formal del \ac{TFG}. L'elaboraci  de la proposta pot implicar l'estudi de conceptes que no s'han cobert en els estudis de grau, la cerca de refer ncies bibliogr fiques addicionals a les subministrades a la proposta del professor o l'avaluaci  d'eines \emph{hardware} i/o \emph{software} que no s'havien utilitzat abans.

La transici  des de la proposta de tema del professor cap a la proposta formal de \ac{TFG} per part de l'alumne pot servir, tamb , per adaptar l'enunciat general d'un problema a un enunciat m s espec fic que reculli les inquietuds de l'estudiant. A m s, la proposta d'un professor o d'un grup de recerca pot acabar derivant en m s d'una proposta de \ac{TFG} si hi ha m s d'un alumne interessat en el tema i  s possible estructurar-la en diversos \ac{TFG}s.

Finalment,  s important recordar que la proposta de \ac{TFG} s'adjuntar  a un contracte docent entre director, alumne i \ac{EPS} i, per tant, ha d'estar consensuada entre les tres parts.


\section{La proposta de tema}

El format de la proposta de tema del professor ser  bastant variable ja que dependr  molt del tipus de projecte concret a desenvolupar i tamb  de l'\emph{estil} de cada professor. En qualsevol cas hauria de contenir informaci  suficient per a que l'alumne pugui tenir una idea prou clara de en qu  consisteix el projecte i quins objectius es persegueixen, de quins coneixements li faran falta per afrontar-ho i de quines s n les refer ncies b siques del tema, que poden servir com a punt de partida per entendre altres refer ncies necess ries per redactar la proposta de \ac{TFG}.

Aix , doncs, alguns elements importants que haurien de quedar reflectits a la proposta de tema s n:
\begin{itemize}
\item Contextualitzaci : La proposta de tema hauria d'incloure una breu introducci  al context del \ac{TFG}, per  suficientment detallada per facilitar qu  l'alumne pugui relacionar la feina a fer amb els coneixements adquirits al llarg dels estudis grau.
\item Definici  d'objectius: S'haurien d'explicitar els objectius que es perseguiran en el projecte. El grau de concreci  dels objectius dependr  del projecte en si: a vegades la proposta de tema ja suggerir  els resultats concrets que es pret n obtenir i en altres ocasions ser  feina de l'alumne definir els objectius concrets del seu \ac{TFG}.
\item Bibliografia: A l'hora de proporcionar una llista de refer ncies bibliogr fiques b siques s'ha de tenir en compta el nivell de coneixements dels alumnes potencialment interessats
    en el tema de \ac{TFG}.  s molt probable que aquesta bibliografia b sica estigui principalment composta de cap tols de llibres i articles introductoris publicats en revistes i congressos.
\item Pre-requisits: La proposta de tema de \ac{TFG} tamb  hauria d'especificar si el \ac{TFG} t  pre-requisits en forma d'assignatures que  s necessari haver cursat (o estar cursant), habilitats (e.g. programaci  de baix nivell, c lcul matem tic) o interessos particulars dels alumnes.
\end{itemize}

La transici  des de proposta de tema publicada per un professor o per un grup de recerca cap a la proposta de \ac{TFG} que ha de redactar l'estudiant, amb el suport del seu supervisor,  s previsible que impliqui les passes seg ents:
\begin{itemize}
\item Immersi  en el tema: L'alumne haur  de ser capa  de relacionar el tema del \ac{TFG} amb les assignatures que ha cursat durant els seus estudis de grau. Aquest proc s ha d'ajudar a l'alumne a prendre consci ncia de quins coneixements b sics li faran falta per dur a terme el \ac{TFG}. La bibliografia b sica proporcionada a la proposta de tema ha de servir per a que l'alumne adquireixi els coneixements fonamentals per contextualitzar el seu treball.
\item Definici  d'objectiu: Un cop assimilats els continguts de les refer ncies introduct ries s'ha de ser capa  de definir de manera ben concreta els objectius del \ac{TFG}. Per aix  pot ser necess ria la consulta d'altres refer ncies m s especialitzades i/o la discussi  d'alguns aspectes del treball amb el supervisor. Aquesta definici  d'objectius tamb  ha de servir, si  s el cas, per incorporar a la proposta els interessos personals del projectista.
\item Programaci  temporal: Tenir objectius ben delimitats i ser conscients dels coneixements necessaris per dur-los a terme ha de permetre a l'alumne fer una programaci  temporal realista del \ac{TFG}. Ser  important comprovar que aquesta programaci  temporal s'ajusta a la c rrega de feina que s'havia previst a la proposta de tema i a la c rrega en cr dits ECTS del \ac{TFG}. Si aquest no  s el cas, ser  necessari tornar a avaluar els objectius del projecte per determinar si aquests s'han d'ampliar o reduir. A l'hora de fer aquesta programaci  temporal  s fonamental que el projectista tingui en compte la c rrega de feina addicional al \ac{TFG} (assignatures, pr ctiques en empresa, \ldots). Els diagrames de PERT o GANT s n eines  tils per realitzar aquesta programaci  temporal.
\end{itemize}
Com ja s'ha mencionat abans, la proposta de tema pot donar lloc a m s d'una proposta de \ac{TFG} si, per exemple, hi ha m s d'un alumne interessat en el tema proposat. En aquest cas ser 
responsabilitat del supervisor verificar que les diferents propostes, encara que parteixin d'un \emph{background} com , tenen objectius concrets ben diferenciats.

\section{Estructura de la proposta de Treball Final de Grau}
En general l'estructura de la proposta de \ac{TFG} inclour :
\begin{itemize}
\item \emph{T tol}: El t tol ha de descriure de la manera m s precisa possible el treball a realitzar. Aquest t tol no ha de coincidir necess riament amb el t tol definitiu que apareixer  a la mem ria.
\item \emph{Paraules clau}: La llista de paraules clau ha d'incloure els conceptes fonamentals que formen la base del \ac{TFG}. En cas de que eventualment s'arribi a disposar d'una base de dades de \ac{TFG}s, aquesta llista ser  molt  til a l'hora de fer-hi cerques en funci  dels temes tractats.
\item \emph{Context}: Aquesta secci  ha de servir per situar el \ac{TFG} dins una determinada  rea de coneixement, per fer una descripci  concisa sobre l'estat de l'art del tema sobre el que tractar  el \ac{TFG} i per aclarir els motius que han portat a la realitzaci  d'aquest treball. Hauria de servir, tamb , per introduir el plantejament general del problema que es pret n abordar. Els conceptes clau del projecte han d'apar ixer en aquesta secci  definits de manera clara per tal de resoldre possibles ambig itats i malentesos entre alumne i supervisor.  s previsible que aquesta secci , juntament amb la dedicada als objectius, constitueixin la base del primer cap tol de la mem ria del \ac{TFG}.
\item \emph{Objectius}: En els objectius s'han de concretar les fites del treball tot indicant les q estions espec fiques adre ades en el \ac{TFG} i els m todes que s'empraran per donar-hi resposta.  s important que siguin objectius mesurables,  s a dir, que el seu assoliment sigui constatable.
\item \emph{Programaci  temporal}: La programaci  temporal descriur  el ritme i ordre en que s'han d'anar realitzant les diferents activitats. El nivell de detall de la planificaci  dependr  del que acordin alumne i supervisor quant a la periodicitat de les reunions des supervisi .
\item \emph{Eines}: Aqu  s'explicitaran aquelles eines \emph{hardware} i/o \emph{software} que es faran servir per dur a terme el \ac{TFG}.
\item \emph{Bibliografia}. La bibliografia inclour  totes les refer ncies bibliogr fiques rellevants en la redacci  de la proposta de \ac{TFG}, tant les proporcionades en la proposta de tema com aquelles que s'hagin pogut descobrir durant els processos de documentaci  i redacci  dels antecedents i objectius de la proposta final.
\end{itemize}
