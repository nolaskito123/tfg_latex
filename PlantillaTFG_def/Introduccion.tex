

\chapter{Introducción}

Els professionals de la comunicaci  tenen molt clar que \emph{la manera com s'explica una hist ria  s tan important com la pr pia historia}. No n'hi ha prou amb el fet de tenir una bona idea, un projecte de negoci extraordinari o uns resultats molt interessants, si no som capa os de presentar-los d'una manera adequada, el m s probable  s que no arribin a bon port.

\emph{Tant en l' mbit acad mic com en el laboral, saber comunicar d'una manera efectiva ens obrir  moltes portes}. En ambd s  mbits, tindrem idees, planejarem projectes o obtindrem resultats que, d'una manera o altra, haurem d'explicar als professors i companys d'una assignatura, als membres d'un tribunal acad mic, als membres del gabinet t cnic d'una empresa, a un possible client o al nostre cap de secci . De l'\emph{impacte} que produeixi la nostra explicaci  (oral o escrita) en dependran, entre d'altres,
\begin{itemize}
 \item \emph{en l' mbit acad mic}: l'acceptaci  a tr mit i la valoraci  d'una tesi, l'obtenci  de finan ament per realitzar un projecte de recerca o la publicaci  d'un article en un congr s o en una revista i,
 \item \emph{en l' mbit laboral}: l'aprovaci  del pressupost per a un projecte de part de la ger ncia de l'empresa, la signatura d'un contracte de serveis amb una altra empresa, la renovaci  del nostre contracte laboral, l'obtenci  de subvencions per engegar un projecte de \acsu{RDI} o la nostra posici  de lideratge.
\end{itemize}
Aix  doncs, atesa la seva rellev ncia per al nostre desenvolupament personal i professional,  s important que ens esforcem per millorar les nostres capacitats de redacci  i presentaci  de treballs cient fics i tecnol gics.

El camp de la comunicaci  cient fico-tecnol gica  s molt ampli, amb tend ncies te riques molt diverses, quantitats ingents de llibres i articles sobre el tema i, fins i tot, programes universitaris de grau i de postgrau. Per raons  bvies, doncs, aquest document no pret n abastar tot aquest camp, ni tan sols substituir la lectura d'altres refer ncies bibliogr fiques molt recomanables. Nom s pret n fer una breu introducci  a les habilitats que s'han de treballar per tal de ser un bon comunicador en qualsevol de les activitats acad miques i professionals pr pies d'un cient fic o d'un enginyer. La millora de les nostres habilitats de comunicaci  es traduir  en la millora dels nostres projectes i de les organitzacions en qu  treballem i, potser m s important, en la millora de la nostra satisfacci  personal i de les nostres carreres acad miques i professionals.

En l' mbit universitari el \acf{TFG} constitueix una part important dels estudis de grau. L'objectiu fonamental d'aquest treball, tot i que sovint requereix estudis addicionals en un camp determinat,  s que els estudiants apliquin els coneixements, les habilitats i les compet ncies adquirides en els seus estudis de grau a la resoluci  d'un problema aplicat. Atesa l'obligatorietat de la presentaci  d'una proposta i d'una mem ria de \acf{TFG} i de la defensa oral d'aquest treball davant un tribunal acad mic, centrarem els continguts d'aquest manual en la introducci  dels principis b sics per a la redacci  i presentaci  d'aquest tipus de treballs. Tanmateix, intentarem que les recomanacions siguin prou generals com perqu  puguin ser f cilment esteses a altres activitats de comunicaci  cient fico-tecnol gica.

Dedicarem el cap tol \ref{instruccions} a fer una petita descripci  de les instruccions generals i l'itinerari a seguir per a la realitzaci  del \ac{TFG}. Al cap tol \ref{mem ria} ens centrarem en els aspectes formals de la redacci  de la proposta i de la mem ria del \ac{TFG}. En el cap tol \ref{presentaci } descriurem els principis b sics i les bones pr ctiques per a la realitzaci  de la defensa oral del \ac{TFG} davant del tribunal. Acabarem amb una secci  que recollir  les conclusions m s importants d'aquest manual.

