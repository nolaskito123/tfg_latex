\chapter{Resum}

La capacitat de redacci  i presentaci  oral de treballs cient fics i tecnol gics  s una de les compet ncies m s importants per al desenvolupament personal i professional d'un cient fic o d'un enginyer. Per tal de millorar aquestes compet ncies, aquest document presenta una breu introducci  a les habilitats que s'han de treballar per tal de ser un bon comunicador en qualsevol de les activitats acad miques i professionals.

At s que en l' mbit universitari la normativa del \ac{TFG} ens obliga a la redacci  d'una proposta i d'una mem ria de \ac{TFG} i a la defensa oral d'aquest treball davant d'un tribunal, en aquest document s'utilitza el \ac{TFG} com a exemple per introduir els principis b sics per a la redacci  i presentaci  de treballs. Tanmateix, les recomanacions que s'hi fan s n prou generals com perqu  puguin ser f cilment esteses a altres activitats de comunicaci  cient fico-tecnol gica.

D'una banda, a partir de la normativa de \acsp{TFG} de l'\ac{EPS} es descriuen les diferents etapes que s'han de superar fins a la defensa oral del \ac{TFG} i de l'altra, s'intenta donar resposta a preguntes del tipus: Qu   s el que fa que la documentaci  o la presentaci  oral d'un treball siguin bones? Quines s n les millors passes a fer per redactar una bona documentaci  o per preparar una bona presentaci ? Quina ha de ser l'estructura global de la documentaci  o de la presentaci ? 