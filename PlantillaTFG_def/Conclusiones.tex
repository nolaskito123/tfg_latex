% !TEX root=ManualTFG.tex

\chapter{Conclusions}

Aquest document he fet palesa la import ncia que tenen les habilitats de redacci  i presentaci  oral de treballs cient fics i tecnol gics per al desenvolupament personal i laboral de qualsevol professional de l' mbit cient fic o de l'enginyeria. Per tal de millorar aquestes compet ncies i at s que en l' mbit universitari la normativa del \ac{TFG} obliga a la redacci  d'una proposta i d'una mem ria de \ac{TFG} i a la defensa oral d'aquest treball davant d'un tribunal, s'ha utilitzat el \ac{TFG} com a exemple per introduir els principis b sics per a la redacci  i presentaci  de treballs cient fico-tecnol gics.

S'han proporcionat tot un seguit de pautes per a la realitzaci  del \ac{TFG}, fent especial esment en la transcend ncia que tenen les tasques de redacci  dels diferents documents que en formen part. S'ha posat de manifest la import ncia de que sigui l'estudiant, a partir de la proposta de tema proporciona per un professor o un grup de recerca, l'encarregat d'elaborar la proposta formal del \ac{TFG}. Aquest proc s d'elaboraci  li proporcionar  una visi  clara, des de l'inici del \ac{TFG}, del problema a resoldre, del seu context i dels objectius concrets de la tasca a realitzar i, a m s, li servir  per visualitzar el full de ruta del \ac{TFG} i li facilitar  el control del progr s en l'execuci  del \ac{TFG} i en l'assoliment dels objectius.

S'han remarcat, tamb , tot un seguit de principis b sics en el proc s d'escriptura cient fica, tals com la precisi , concisi , claredat, coher ncia i adequaci  a l'audi ncia, que fan que una mem ria de \ac{TFG} sigui d'alta qualitat i s'han donat les pautes a seguir en els processos de focalitzaci , elaboraci  de l'esb s i redacci  de la mem ria de \ac{TFG} per tal de ser fidels a aquests principis.

Finalment, s'ha fet un descripci  curosa del proc s a seguir quan es vol reflectir en una presentaci  oral la feina realitzada en el \ac{TFG}, tenint en compte tant els aspectes formals i estructurals del document de la presentaci  com els aspectes de comunicaci  oral.
