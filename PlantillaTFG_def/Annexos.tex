%!TeX root=ManualTFG.tex

\chapter{Format final}

\section{Suport electr nic}
Cal utilitzar un CD/DVD amb una portada/cap alera normalitzada seguint la plantilla que trobareu a la web de la EPS a l'apartat de TFGs. Dins el CD/DVD s'ha d'incloure:
\begin{itemize}
  \item Un arxiu incloent nomes el resum del TFG amb format pfd.
  \item Un arxiu amb la mem ria amb format pdf.
  \item Si escau, un arxiu per cada annex/appendix en format pdf.
  \item Si escau, un arxiu amb els pl nols en format pdf.
\end{itemize}

\section{Paper, impressi  i enquadernaci }

\subsection{Paper}

Cal utilitzar paper mida DIN A4 vertical (210 x 297 mm), el qual, a m s de ser l'est ndard m s generalitzat,  s el format predeterminat de
la majoria de processadors de textos. La mem ria del TFG ha de tenir com a m xim una extensi  de 4 cares*N ECTS. Si la mida del treball  s superior el m s recomanable  s resumir alguns apartats o, en casos concrets, es pot optar per distribuir informaci  cap als annexos. 

\subsection{Impressi  a dues cares}

La presentaci  del document ha de ser a dues cares a partir de la introducci  i fins al final del document. 
\subsection{Enquadernaci }

Cal realitzar l'enquadernaci  amb espiral negra. Les tapes superior i inferior seran de pl stic transparent i negre, respectivament.

\section{Format del document}
A la web de l'escola polit cnica es proveeixen unes plantilles


recomanades tant en  \LaTeX  com en Word i Open Office per a la


realitzaci  de la mem ria del treball fi de grau. En cas de no



utilitzar aquestes plantilles l'estructura final de la mem ria ha de ser an loga a la de la plantilla recomanada. En particular s'han de seguir les seg ents instruccions de format:

\begin{itemize}
\item Configuraci  de la p gina: Totes les p gines han d'incloure com a m nim el n  de p gina a la cap alera o el peu i  s recomanable incloure tamb  el t tol del cap tol en curs. Els marges de p gina han de ser de mida semblant als de la plantilla.
\item par graf: interlineat simple, alineaci  justificada.
\item Mida de lletra: 
\begin{itemize}
\item text 11 punts
\item peus de figures i taules 9 punts
\end{itemize}
\item La llista d'acr nims (si escau) i la bibliografia han de seguir el mateix format que el de les plantilles recomanades.
\end{itemize}

\subsection{la plantilla de \LaTeX }
En el cas d'utilitzar la plantilla  \LaTeX es important llegir tots les instruccions d'us que es troben comentades a l'inici de la plantilla. A m s,  s important mencionar que aquesta plantilla defineix la portada, els marges, la tipografia, estils i espaiat de tots els elements pel treball de final de
grau. Una vegada es compila el document, \LaTeX\  adequar  el text al format definit en la plantilla. A m s, \LaTeX\ realitzar  de forma
autom tica tot un seguit de funcions que us facilitaran la feina amb la mem ria, com per exemple: numerar els cap tols, seccions, i
sub-seccions; inserir els encap alaments i n meros de p gina; crear de forma autom tica els  ndexs de continguts, figures i taules \dots
Per tant, l'usuari nom s es preocupar  del text que est  introduint, no ser  necessari pensar en cap moment en l'aspecte final del
document. Ser  \LaTeX\ qui aplicar  el format de la plantilla al teu text.

Aquesta plantilla est  fonamentada en la classe \texttt{memoir}. Aix  presenta l'avantatge de que com que aquest format inclou autom ticament altres \emph{packages}, per exemple \texttt{booktabs},
\texttt{array}, \texttt{tabularx}, etc., no caldr  carregar-los en el pre mbul del vostre document. Aix  tamb  significa que totes les comandes de \texttt{memoir} estan a la vostra disposici  per editar la mem ria, per tant,  s molt recomanable consultar el seu manual~\cite{Wil10}.

Amb el format de plantilla que s'ha definit nom s s'enumeraran 3 nivells de profunditat, a part dels cap tols. Per definir-los s'utilitzaran les
seg ents expressions de \LaTeX:

\begin{verbatim}
\chapter{Nom del cap tol}
\section{Nom de la secci }
\subsection{Nom de la sub-secci }
\subsubsection{Nom de la sub-sub-secci }
\end{verbatim}

\section{F rmules, figures i taules}
\subsection{F rmules}

El format de les f rmules es troba definit en la plantilla i \LaTeX\ l'aplica cada vegada que es compila el document.

Per escriure f rmules en \LaTeX\ s'hauran d'utilitzar les expressions adients. Aqu  es presenta un exemple de codi:
\begin{verbatim}
\begin{equation}\label{NomEq}
\zeta= m \sum _{i=0}^{N} \left( \frac{\beta}{\sigma _i \lambda_ j}
\right)^{2} \cos (2\pi f_i)
\end{equation}
\end{verbatim}
que produeix el seg ent resultat:
\begin{equation}\label{NomEq}
\zeta= m \sum _{i=0}^{N} \left( \frac{\beta}{\sigma _i \lambda_ j}\right)^{2} \cos (2\pi f_i).
\end{equation}

Citar l'expressi  anterior  s tant senzill com fer:
\begin{verbatim}
L'equaci  \ref{NomEq} determina $\ldots$
\end{verbatim}
L'equaci  \ref{NomEq} determina $\ldots$

\subsection{Figures}

A continuaci  es mostra el codi \LaTeX\ per incloure una figura continguda en un fitxer.

\begin{verbatim}
\begin{figure}[htb]
\begin{center}
\includegraphics[width=0.2\textwidth]{./LogoUIB.jpg}
\caption{Exemple de figura}
\label{NomFig}
\end{center}
\end{figure}
\end{verbatim}
El resultat es pot veure a la Fig.~\ref{NomFig}.

\begin{figure}[htb]
\begin{center}
\includegraphics[width=0.2\textwidth]{./LogoUIB.jpg}
\caption{Exemple de figura}
\label{NomFig}
\end{center}
\end{figure}

Es pot modificar la variable \texttt{width} per ajustar l'amplada de la figura com m s ens convingui. Teniu en compte que la variable
\texttt{\textbackslash textwidth} guarda el valor de l'amplada del text dins la p gina i, per tant,  s una bona refer ncia per delimitar amplades de figura. Aix  doncs, la figura \ref{NomFig} ocupa la meitat de l'amplada del text en una p gina. El format final de la figura est  definit per la
plantilla i \LaTeX\ s'encarrega de presentar-la de forma convenient.

\subsection{Taules}

Les taules definides en \LaTeX\ s'enumeren autom ticament i el format segueix les definicions especificades en la plantilla.

Seguidament, a mode d'exemple, es presenta les expressions \LaTeX\  per a crear
la taula \ref{NomTaula} que apareix m s avall:
\begin{verbatim}
\begin{tabular}{@{}llS@{}} 
\toprule
\multicolumn{2}{c}{Cotxes} \\ 
\cmidrule(r){1-2}
{Posici } & {Descripci } & {Velocitat m xima}\\
 & &  \multicolumn{1}{s}{(\kilo\meter\per\second)} \\ 
\midrule
1 & Vermell & 120 \\
2 & Blau & 80.1 \\
3 & Verd & 92.50 \\
4 & Blanc & 33.33 \\
5 & Negre & 56.3 \\ 
\bottomrule
\end{tabular}
\caption{Exemple de taula} \label{NomTaula}
\end{table}
\end{verbatim}
\begin{table}
\centering
\begin{tabular}{@{}llS@{}} \toprule
\multicolumn{2}{c}{\textbf{Cotxes}} \\ 
\cmidrule(r){1-2}
{\textbf{Posici }} & {\textbf{Descripci }} & {\textbf{Velocitat m xima}}\\
 & &  \multicolumn{1}{s}{(\kilo\meter\per\second)} \\ \midrule
1 & Vermell & 120 \\
2 & Blau & 80.1 \\
3 & Verd & 92.50 \\
4 & Blanc & 33.33 \\
5 & Negre & 56.3 \\ \bottomrule
\end{tabular}
\caption{Exemple de taula} \label{NomTaula}
\end{table}

L'entorn \texttt{tabular} que ofereix \LaTeX\   s molt complet i permet crear
multitud de taules diferents, tot i que alhora  s bastant complexe. No s n les
intencions del present document descriure la sintaxis i el format d'aquest
tipus d'entorn. Es poden trobar molt f cilment \emph{tutorials} o altres informacions
per aprendre a utilitzar de forma adient aquesta sintaxis o qualsevol altra de
\LaTeX.  s bastant recomanable llegir la documentaci  del \emph{package}
\texttt{booktabs}\footnote{No cal incloure la comanda  \texttt{\textbackslash usepackage\{booktabs\}} dins el document perqu  la classe ja ho fa.}~\cite{Fea05} on s'introdueixen una s rie de comandes per a poder realitzar
taules de m s qualitat com la de l'exemple, tamb  es defineixen quines han de
ser les pautes per fer una taula d'aspecte formal. En aquest exemple concret tamb  s'han usat les columnes \texttt{S} i \texttt{s} que ofereix el paquet \texttt{siunitx}~\cite{Wri12}. Un efecte similar es podria aconseguir amb les columnes de tipus \texttt{D} que inclou \texttt{memoir}~\cite[Cap. 11]{Wil10}.

Cal fixar-se en que \LaTeX\ insereix les figures i taules sempre al principi de p gina. Per tant, no cal preocupar-se per la seva posici 
dintre del document s'insereixen sempre en la mateixa posici  de forma autom tica.

\section{Bibliografia}

A la bibliografia s'han de llistar conjuntament llibres i articles de revistes.
Citar una refer ncia bibliogr fica  s tant f cil com fer:
\begin{verbatim}
\cite{bib1}, \cite{bib2}, \cite{bib3}
\end{verbatim}
per citar la refer ncia \cite{bib1}, \cite{bib2}, \cite{bib3}. 

El format de la bibliografia es genera autom ticament.

\section{Acr nims}

Per exemple, un acr nim ben conegut  s  \ac{IP}.
% Aquesta sentencia ens permetr  generar una entrada a la llista d'acr nims.
% En la llista d'acr nims es definiran cada un dels acronims i mitjan ant l'expressi  anterior podrem referenciar-los.
