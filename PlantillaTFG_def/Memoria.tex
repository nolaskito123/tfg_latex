\chapter{La mem ria del treball de final de grau}\label{mem ria}
\section{Principis b sics}

\emph{Qu   s el que fa que una mem ria de TFG sigui bona?}  bviament, la resposta a aquesta pregunta pot ser molt complexa, per  hi ha molts autors (vegeu, per exemple, \cite{Perelman01,Malvar08} i els treballs que aquests referencien) que coincideixen en que un informe t cnic o cient fic ha de reunir les qualitats seg ents:
\begin{itemize}
   \item \emph{precisi }: aquesta qualitat es fonamenta en tres aspectes complementaris:
   \begin{enumerate}

   \item \emph{precisi  del document}, que fa refer ncia a que aquest s'ha de concentrar de forma precisa en el tema que el defineix i n'ha de fer un tractament correcte, basat en evid ncies cient ficament demostrables, i amb un nivell de detall apropiat, ni massa general ni excessivament restringit.

   \item \emph{precisi  estil stica}, que nom s  s possible si es fa un  s cur s del llenguatge per tal d'expressar el significat. La precisi  en l' s del llenguatge es fonamenta en el rigor en la utilitzaci  de les paraules, en una estructura correcta de les frases i en un  s adequat dels signes de puntuaci .

   \item \emph{precisi  t cnica}, que es fonamenta en un bon coneixement t cnic de la mat ria i del seu vocabulari.
   \end{enumerate}

   \item \emph{concisi }: molt relacionada amb la precisi  del document, la concisi  es fonamenta en la focalitzaci ,  s a dir, en la reducci  de l'abast del document a l' mbit estricte de la q esti  que es vol tractar.  s molt f cil caure en la temptaci  d'incloure en la mem ria materials que potser s n molt rellevants en el camp en qu  es desenvolupa el treball, per  que no ho s n en absolut per comunicar de manera efectiva les aportacions concretes del treball en aquest camp.
       %Com veurem m s endavant, la definici  precisa i detallada de l' ndex de la mem ria  s una de les estrat gies m s  tils per controlar la longitud i l'abast del document. Tamb   s important identificar i eliminar els materials (frases, par grafs, seccions, \ldots) que no s n necessaris per recolzar o evidenciar les nostres aportacions. Les figures i les taules solen contribuir a la concisi  del document perqu  redueixen la quantitat de prosa necess ria per descriure objectes i processos, resumeixen dades i serveixen per demostrar relacions.
%
   \item \emph{claredat}: aquesta qualitat, que fa refer ncia a la capacitat de qui escriu per facilitar la comprensi  del document a qui llegeix,  s especialment important en l'escriptura cient fico-t cnica. Els vocabularis especialitzats, els desenvolupaments matem tics o els esquemes conceptuals complexos poden dificultar molt ssim la comprensi  d'algunes explicacions t cniques, fins i tot quan aquestes han estat redactades per escriptors especialitzats i s n processades per lectors experts.

   \item \emph{coher ncia (organitzaci  i estructura)}: un document  s coherent si el material que presenta est  organitzat de manera l gica i consistent i la seva estructura proporciona al lector un cam  f cil per a la seva comprensi . La coher ncia es valora de forma molt especial en ci ncia i tecnologia degut a la inherent complexitat dels temes que es tracten.


   \item \emph{adequaci  a l'audi ncia}:  s convenient que el document s'adeq i als objectius que l'autor s'ha marcat a l'hora d'escriure'l, per , sobretot,  s molt important que s'adeq i a les necessitats dels possibles lectors (tutors, membres del tribunal, altres alumnes que treballin en temes semblants, \ldots) i al context, objectius i convencions (forma i estil) de la instituci  en qu  es presenta.
\end{itemize}

Tal com ens recorda H. S. Malvar a \cite{Malvar08}, l'error m s freq ent que podem cometre a l'hora d'escriure, especialment en el cas de l'escriptura cient fica,  s no saber posar-nos al lloc del lector. Aix  doncs, la recomanaci  fonamental per evitar errors a l'hora d'escriure la mem ria  s que cada cop que afegim una nova argumentaci , una nova figura o una nova taula, pensem en els possibles lectors d'aquesta informaci  i la revisem per tal de garantir que  s precisa, clara, concisa, coherent i adequada a l'audi ncia.

\section{Bones pr ctiques}

\emph{Quines s n les millors passes a fer per escriure una bona mem ria de TFG?} La major part dels treballs de final de grau es desenvolupen en tres etapes, una primera etapa en qu  es duu a terme una revisi  de les refer ncies bibliogr fiques m s rellevants sobre el context general i l' mbit particular del projecte, una segona etapa en qu  es realitzen estudis anal tics, simulacions, implementacions reals, \ldots i, finalment, una tercera etapa en qu  es redacta la mem ria. Tanmateix, aquestes tres etapes no s n estrictament consecutives i, habitualment, treuen profit d'una retroalimentaci  sistem tica entre elles. En aquesta secci , tot i ser molt conscients d'aquesta interdepend ncia, ens centrarem en l'etapa de redacci  de la mem ria i presentarem una s rie de recomanacions a seguir en els processos de \emph{focalitzaci }, \emph{elaboraci  de l'esb s} i \emph{redacci }.

\subsection{Focalitzaci } Aquest proc s aborda de ple les qualitats de precisi  del document, concisi  i adequaci  a l'audi ncia. La focalitzaci  consisteix en la reducci  de l'abast del document a l' mbit estricte que el defineix, en la selecci  acurada dels temes sobre els que ha de tractar i en l'adequaci  dels continguts i profunditat de tractament d'aquests a l'audi ncia concreta a la que es vulgui dirigir. Per focalitzar un document, algunes de les passes que ens podrien servir serien:
\begin{enumerate}
   \item Fer una llista completa dels temes o paraules clau que defineixen el nostre treball.

   \item Si encara no ho hem fet, rec rrer a les refer ncies bibliogr fiques que calgui per tal de tenir una visi  prou acurada del paper que juga cadascun d'aquests temes en el context general i en l' mbit particular del tema que ens ocupa.

   \item A partir de la llista de temes i de la informaci  disponible sobre cadascun d'aquests, respondre a preguntes del tipus:
   \begin{itemize}
          \item  s necessari que tracti aquest tema en la mem ria?
          \item Es perdran els lectors de la mem ria si no tracto aquest aspecte?
          \item Puc deixar de parlar sobre aquest tema sense perjudicar l'objectiu global del meu projecte?
          \item  s excessivament general aquest tema o  s excessivament espec fic?
          \item N'hi ha prou amb un tractament general acompanyat de refer ncies bibliogr fiques o seria millor que el lector dispos s d'una visi  m s detallada per tal de poder seguir els raonaments posteriors?
          \item  s necessari introduir coneixements previs per poder tractar alguns dels temes de la llista?
   \end{itemize}
   Aquest tipus de preguntes actuaran com a filtres que ens permetran eliminar temes innecessaris, afegir-ne de necessaris o preveure la profunditat amb la que s'ha de tractar cadascun dels temes seleccionats. En haver acabat disposarem d'una bona eina per passar a l'etapa d'elaboraci  del primer esb s de la mem ria.
\end{enumerate}

\subsection{Elaboraci  de l'esb s} Un cop hem focalitzat l' mbit estricte del treball i, per tant, disposem de la llista de temes seleccionats i tenim una idea general de la profunditat amb qu  els volem tractar, una bona manera de facilitar el proc s de redacci  consisteix en la preparaci  d'un esb s detallat del nostre treball. Es comen a amb una taula de continguts que cont  un llistat dels t tols provisionals dels cap tols, seccions i subseccions que es volen incloure en la primera versi  de la mem ria. Despr s, per cadascun dels cap tols, seccions i subseccions es redacten una s rie de frases curtes que descriuen, de forma ordenada, els aspectes clau de tots i cadascun dels continguts que s'hi tractaran i, si es creu convenient, les fonts en les que ens podem recolzar en el proc s de redacci . En aquestes etapes continua sent especialment important tenir en ment les qualitats de precisi , concisi  i adequaci  a l'audi ncia de la nostra proposta, per  tamb  hi jugar  un paper especialment rellevant la qualitat de coher ncia. Paga la pena dedicar temps i esfor  en l'elaboraci  d'aquest primer esb s i revisar-lo tantes vegades com sigui necessari fins a estar-ne totalment conven uts.

Un cop tenim l'esb s provisional del document  s el moment de fer-ne una revisi  acurada amb l'ajut del nostre tutor. Aquesta revisi  ens hauria de permetre, entre d'altres coses, eliminar el material innecessari, afegir el material o argumentacions que se'ns hagin pogut oblidar i reestructurar els continguts per tal d'incrementar el nivell de coher ncia del document.  s m s eficient i menys dolor s prendre aquestes decisions en les fases inicials del proc s de redacci , que haver-les de prendre un cop ja s'ha escrit molt de material que al final s'ha de descartar. A partir d'aquest primer esb s ens resultar  m s senzill visualitzar l'estructura global de la mem ria i saber, en cada moment, quin  s l'estat del proc s de redacci . Ens permetr  determinar si el treball realitzat (simulacions, implementacions, an lisis, \ldots), els resultats obtinguts i les refer ncies bibliogr fiques consultades s n suficients o si  s necessari aprofundir en algun aspecte concret. Ens ajudar , tamb , a planificar la nostra feina en funci  del temps disponible.

\subsection{Redacci } L'estrat gia de redacci  a seguir a partir d'aquest primer esb s no ha de ser necess riament lineal,  s a dir, no ha de comen ar necess riament amb la redacci  del cap tol d'introducci  i acabar amb la del cap tol de conclusions. De fet, tot i que  s molt important partir d'una idea molt clara dels continguts del cap tol d'introducci , at s que aquest  s el que proporciona una visi  global del context i l'abast del treball, en la majoria dels casos ens pot resultar m s senzill comen ar amb la redacci  dels cap tols corresponents al desenvolupament del treball i acabar amb la redacci  de la introducci  i les conclusions.  s l'anomenada estrat gia \emph{de dintre cap a fora}.

 s molt possible que, a mesura que anem redactant aquesta primera versi  de la mem ria, tamb  descobrim que  s necessari fer addicions, supressions i altres esmenes sobre l'esb s original a fi d'assolir-ne la versi  definitiva. En qualsevol cas, at s que les frases curtes que hem utilitzat apunten, de forma ordenada i coherent, els aspectes clau de tots i cadascun dels continguts dels cap tols, seccions i subseccions de la mem ria, si ens dediquem a desenvolupar-les en forma de par grafs, recorrent, sempre que sigui necessari, a l'ajut de figures, taules, desenvolupaments matem tiques, \ldots, al final obtindrem la primera versi  de les diferents parts de la mem ria.

La primera versi  d'una secci  o d'un cap tol  s la primera passa cap a la versi  definitiva, per  no ens ha de fer mandra revisar-la i reescriure-la tants cops com sigui necessari per tal de millorar-ne les qualitats de precisi , claredat, concisi , coher ncia i adequaci . Aquest proc s de revisi  s'ha de fer a consci ncia, intentant descobrir paraules, frases, taules, figures o, fins i tot, par grafs que no contribueixen a que el text sigui m s prec s o m s conc s o m s clar, procurant, tamb , millorar l'organitzaci  i l'estructura dels diferents elements que conformen el text i, no menys important, mirant d'aconseguir la precisi  estil stica a trav s de la correcci  ortogr fica i gramatical.

La paraula clau  s revisi  i, com no podia ser d'altra manera, un cop donem per acabada la primera versi  d'un cap tol  s el moment de que el nostre tutor tamb  la revisi. At s que ja hav em mantingut reunions amb el tutor per discutir l'esb s de la mem ria,  s poc probable que aquesta revisi  suposi canvis substancials en l'estructura general del cap tol. Tanmateix, hi pot haver aspectes del treball que nom s s'hagin tractat de manera m s o manco informal i, per tant, la lectura d'aquesta versi  pot ser el primer contacte formal del tutor amb alguns dels plantejaments te rics o pr ctics realitzats per l'alumne. Aix  doncs, hi ha la possibilitat de que es detectin mancances o errors que suposin el replantejament d'alguns apartats.  s molt important que considerem tots els suggeriments que ens faci el tutor i que els utilitzem per millorar la seg ent versi  de la mem ria. Si s'han produ t replantejaments d'algunes parts del treball pot ser necess ria una segona revisi  per part del tutor.

Quan el tutor hagi revisat tots els cap tols de la mem ria i nosaltres h gim realitzat tots els canvis oportuns, nom s restar  portar a terme una revisi  global de la mem ria per tal d'obtenir-ne la versi  definitiva.

\section{Estructura del TFG}

Tal com ens recorda el professor Valiente \cite{Valiente96}, organismes internacionals d'estandarditzaci  com l'ANSI (\emph{American National Standards Institute}) o la ISO (\emph{International Organization for Standardization}) prescriuen sistemes est ndard per a l'organitzaci  dels treballs cient fics que contenen quatre parts fonamentals: introducci , desenvolupament, resultats i conclusions. Aquestes parts fonamentals d'un treball cient fic es complementen amb altres components com la portada, la taula de continguts, les llistes de figures, taules i acr nims, el resum, els agra ments, els ap ndixs o les refer ncies bibliogr fiques.A continuaci  es descriu el contingut de cada una d'aquestes parts.

 En el cas de projectes t cnics (projectes de rehabilitaci  d'edificis, projectes d'ajardinament...) es seguiran les especificacions de la norma UNE 157001:2002 i les recomanacions de les assignatures de projectes dels plans d'estudis corresponents a l'hora d'estructurar la mem ria del TFG.


\subsection{Portada}

La portada actua com a element de presentaci  i identificaci  del TFG. Les dades que s'hi han de fer constar poden variar en funci  del tipus de treball, per  n'hi ha algunes que s n fonamentals: t tol, autors, tutors i, si s'escau, tutors, departament, universitat, t tol acad mic al qual s'opta i data de presentaci . Despr s de la portada s'acostumen a deixar un o m s fulls en blanc de cortesia.

El t tol  s sens dubte una part molt important de la mem ria del TFG.  s el primer lligam que s'estableix entre el TFG i el lector i, per tant, s'ha de ser molt cur s a l'hora de seleccionar les paraules i les frases que donaran forma al t tol. Un bon t tol  s aquell que descriu el contingut del TFG de manera precisa i amb el menor nombre de paraules possible. Una bona estrat gia a l'hora d'escriure el t tol del TFG  s partir d'una llista de paraules clau i tractar de trobar quines d'aquestes s n fonamentals a l'hora de descriure la nostra aportaci  i quin ha de ser l'ordre i l'associaci  entre aquestes paraules clau.  bviament, a mesura que anem escrivint la mem ria podem anar manejant una s rie d'alternatives que puguin donar lloc a diferents t tols i deixar que amb el temps alguna d'elles es vagi imposant sobre les altres.

\subsection{Taula de continguts o  ndex}

La taula de continguts o  ndex  s un llistat dels t tols dels diferents cap tols, seccions i subseccions del document amb indicaci  dels n meros de p gina en qu  apareixen. Per tant, la taula de continguts no nom s ajuda als lectors a cercar els diferents temes tractats en la mem ria, sin  que tamb  serveix com a esb s de l'estructura de la mem ria i ofereix una visi  general del document als lectors potencials.  bviament, les taules de continguts m s  tils es componen de t tols de caire descriptiu.

\subsection{Llista de figures}

Els lectors utilitzen la llista de figures per localitzar la informaci  visual en la mem ria. La llista de figures relaciona els t tols o llegendes dels recursos visuals (figures, dibuixos, fotografies, \ldots) amb la seva ubicaci  dintre de la mem ria.  s important que les llegendes de les figures siguin descriptives i que estiguin numerades de manera consecutiva.

\subsection{Llista de taules}

La llista de taules proporciona les llegendes i la localitzaci  de totes les taules que apareixen a la mem ria. De la mateixa manera que els t tols de figura, els t tols o llegendes de les taules s'han de numerar consecutivament en l'ordre en qu  apareixen al document.

\subsection{Llista d'acr nims}

Hi ha documents que utilitzen una quantitat important de termes nous, o molts acr nims i abreviacions. En aquests casos es pot facilitar la lectura de la mem ria si s'inclou una llista de nomenclatura o una llista d'acr nims just despr s de les llistes de figures i/o taules. Un dels efectes secundaris interessants de les llistes de nomenclatura o d'acr nims  s que ajuden a l'autor del document a utilitzar la terminologia d'una manera coherent.

\subsection{Resum}

El resum  s una breu declaraci , generalment entre 250 i 500 paraules, que proporciona al lector una sinopsi del problema, el m tode, els resultats i les conclusions de la mem ria. Els resums s'han de poder llegir de manera totalment independent de les altres parts de la mem ria i, per tant, no s'hi han d'utilitzar acr nims sense definir-los i tampoc s'hi han d'utilitzar refer ncies bibliogr fiques. Els resums s n extremadament  tils per aquelles persones que volen tenir una imatge general del contingut de la mem ria abans de llegir el document principal. At s que hi pot haver lectors potencials que utilitzin el resum per decidir si han de continuar llegint la mem ria o no, cal no menystenir la import ncia d'una bona redacci  d'aquest apartat del document. Els resums poden estalviar una immensa quantitat de temps als possibles lectors.

El resum ha d'incloure, com a m nim, els seg ents elements:
\begin{itemize}
   \item Definici  abreujada del problema o tema principal del TFG.

   \item Exposici  del m tode utilitzat per resoldre el problema.

   \item Comentaris sobre els principals resultats, aportacions i possibles aplicacions del treball.

   \item Conclusions m s importants del treball
\end{itemize}

\subsection{Agra ments}

A vegades s'inclou una secci  d'agra ments en els preliminars de la mem ria per tal de donar cr dit a l'assist ncia rebuda de part de persones i/o institucions. Els tutors, els t cnics de laboratori o els companys de feina que ens han assessorat o ens han donat suport s n, tots ells, candidats a apar ixer al cap tol d'agra ments.

\subsection{Introducci }

Si hem decidit utilitzar l'estrat gia \emph{de dintre cap a fora}, despr s de redactar els cap tols corresponents al desenvolupament del treball estarem en disposici  d'enllestir la redacci  de la introducci  i les conclusions.

La introducci  hauria de servir per donar, de forma descriptiva i f cil d'entendre, una visi  global del context i l'abast del treball. De fet, segons Booth \emph{et al.} \cite{Booth08}, el patr  com  que cerquen els lectors en qualsevol introducci  est  format per tres elements:
\begin{itemize}
   \item Contextualitzaci : es tracta d'explicitar el context en el que s'emmarca el treball, d'establir la base comuna de coneixements sobre el tema que es tractar  en la mem ria del TFG, de garantir que els possibles lectors comparteixen amb l'autor de la mem ria el conjunt de factors que els permetran interpretar adequadament els seus enunciats i raonaments.

       El context d'un TFG podria ser, per exemple, el m n de les xarxes de comunicacions m bils de quarta generaci  (4G), amb capes f siques basades en l' s de m ltiples antenes en transmissi  i en recepci  (MIMO -- \emph{Multiple-Input Multiple-Output}), t cniques de codificaci  i modulaci  adaptatives (AMC -- \emph{Adaptive Modulation and Coding}) i estrat gies d'acc s m ltiple basades en l' s de transmissi  multiportadora (OFDMA -- \emph{Orthogonal Frequency Division Multiple Access}). En aquest cas seria adequat parlar sobre l'estat actual del desenvolupament dels est ndards 4G, de les caracter stiques generals de les tecnologies MIMO, AMC i OFDM(A) i de la seva adequaci  als sistemes 4G. Depenent de l' mbit d'aplicaci  del problema a tractar podria ser adequat aprofundir, per exemple, en la descripci  de l'estat actual de les xarxes de comunicacions ce\l.lulars i de les caracter stiques de les estacions base i/o de les estacions repetidores o en la descripci  de l'estat actual de les xarxes d' rea local sense fils i de les possibles estrat gies de cooperaci  entre punts d'acc s.

   \item Definici  del problema: un cop establert el context del treball,  s l'hora de definir amb precisi  el problema que es tractar  en el TFG i de justificar la seva import ncia dintre del context en el que s'emmarca. Es tracta, tamb , de proporcionar una visi  general, per  concisa, dels antecedents del problema, de les publicacions m s rellevants sobre el tema, de les virtuts i mancances dels plantejaments i solucions aportades per altres autors.

       Dintre del context de l'exemple anterior, un possible problema a resoldre en un TFG podria ser, atesa la necessitat de gestionar els recursos disponibles d'una manera eficient, el de l'assignaci   ptima de pot ncia, subportadores i modes de transmissi  (codificaci  de canal i modulaci ) per tal de garantir una taxa de transmissi  global m xima amb restriccions sobre la qualitat de servei proporcionada a les aplicacions dels diferents usuaris del sistema.

   \item Resposta al problema: despr s de definir el problema, el m s l gic  s presentar al lector de la mem ria la nostra proposta per solucionar-lo, els nostres objectius. Es tracta d'explicitar el m tode seleccionat per solucionar el problema plantejat anteriorment, tot justificant aquesta selecci . Tamb  pot ser adequat avan ar, tot i que de manera concisa, els resultats principals del TFG i les possibles conclusions que es desprenen dels resultats obtinguts.

       En el problema de l'exemple anterior, una possible resposta podria consistir en el desenvolupament d'algorismes d'optimitzaci  dual, en l' s de la teoria de jocs o, per posar-ne un altre exemple, en l' s d'aprenentatge estad stic (\emph{machine learning}). En cadascun d'aquests casos haur em de justificar la tria feta i, si ens sembl s adient, podr em parlar de quins s n els resultats que mostrarem al lector en el desenvolupament de la mem ria del TFG.

\end{itemize}


Avui en dia, tant en la universitat com en l'empresa,  s molt habitual que el TFG formi part de projectes de recerca m s amplis, de manera que pot ser dif cil per al lector discernir quan  s que l'autor descriu el seu treball personal i quan  s que descriu una tasca realitzada per altres membres del grup de recerca. En aquests casos,  s molt important que l'autor expliciti el millor possible quin ha estat exactament el seu paper dintre d'aquest projecte general.

Tot i que hi ha autors que no ho recomanen, una manera habitual d'acabar una introducci  consisteix en presentar un resum de l'estructura de la mem ria, avan ant al lector quins s n els continguts dels diferents cap tols del TFG.


\subsection{Desenvolupament}

Aquesta part de la mem ria, que es pot estructurar en diversos cap tols, tracta sobre la pr pia realitzaci  del treball i descriu el que s'ha fet, com s'ha fet, per qu  s'ha fet d'aquesta manera i no d'una altra, quins materials o eines s'han utilitzat o s'han hagut de desenvolupar, quina metodologia de treball i de validaci  s'ha seguit, \ldots

L'estructura, organitzaci  i contingut d'aquesta part de la mem ria depenen en gran mesura del tipus de TFG: emp rics, estudis de casos, metodol gics, te rics, \ldots\ Tanmateix, el principi b sic ha de ser proporcionar informaci  suficient perqu  un lector ben informat pugui comprendre, reproduir i verificar els experiments o els desenvolupaments te rics, tot evitant la simple repetici  enciclop dica de coneixements que es poden trobar a llibres, articles o altres documents de refer ncia. Per exemple, quan la mem ria del TFG comenci amb els fonaments te rics del problema plantejat a la introducci , el seu tractament, necess riament sint tic, ha de mostrar l'elaboraci  personal de la informaci  manejada i s'ha de tenir molta cura de no caure en la simple c pia dels autors referenciats. A partir d'aquesta elaboraci  personal s'entendr  el fil argumental seguit per l'autor a l'hora d'arribar a la resoluci  del problema plantejat.

\subsection{Resultats i discussi }

Els resultats obtinguts en el TFG constitueixen la nostra contribuci  al coneixement cient fic. Aix , doncs, aquesta part de la mem ria ha de descriure tota la informaci  generada en el desenvolupament del TFG. No n'hi haur  prou amb presentar les dades juntament amb les estimacions sobre la seva precisi , tamb  ser  necessari interpretar-les i situar-les en context comparant-les amb les obtingudes per altres autors o utilitzant altres m todes proposats a la literatura.

En els cap tols dedicats a la presentaci  i discussi  de resultats  s habitual utilitzar figures i taules per tal de mostrar les dades d'una manera efectiva.  s important parar molt d'esment en l'elaboraci  tant de les figures com de les taules i, tamb , que en el text fem refer ncia expl cita als resultats que hi presentem. Si el TFG ha produ t una gran quantitat de dades potser no cal presentar-les totes en els cap tols de resultats i el m s adequat  s fer-ne una selecci  acurada que ens permeti complir amb el prop sit d'extreure'n i fonamentar de forma rigorosa les conclusions del TFG i poder-ne fer una presentaci  adequada als lectors. De fet, si es considera oport , les dades que no apareixin en aquests cap tols de resultats es poden fer avinents als possibles lectors en els ap ndixs.

\subsection{Conclusions}

Tot i que el cap tol de conclusions d'un TFG pot comen ar amb un resum del context i de la definici  del problema i passar despr s a analitzar la import ncia del treball realitzat i dels resultats obtinguts, les conclusions no s'han de limitar a tornar exposar el que ja s'ha presentat en els cap tols anteriors i, a m s, no s'ha de caure en la trampa de repetir el mateix que es va dir a la introducci  \cite{Pierson97}. Per tant, el resum del context i de la definici  del problema ha de ser molt breu i ens hem de concentrar en la interpretaci  dels resultats i en la identificaci  de les nostres contribucions. Les conclusions han de donar resposta a preguntes del tipus: Quines implicacions te riques i/o pr ctiques pot tenir el meu treball? Quin valor afegit suposa aquest treball dintre del corpus de coneixements de la meva disciplina? Qu   s el que coneixem ara que no se sabia al principi d'aquest treball? Quines mancances i quina utilitat tenen els resultats obtinguts? Qu  es pot fer a partir d'aquests resultats? Quines portes hem tancat i quines hem deixat obertes? Quines recomanacions podem fer als que vulguin continuar en aquesta l nia?

Aix , doncs, el cap tol de conclusions hauria de:
\begin{itemize}
   \item Recordar de manera concisa el context i la definici  del problema i dels possibles objectius marcats en la introducci  del TFG.

   \item Argumentar les principals conclusions del TFG i establir les possibles implicacions te riques i les possibles aplicacions pr ctiques dels resultats obtinguts.

   \item Remarcar qu   s el que ha quedat i el que no ha quedat demostrat en la mem ria del TFG.

   \item Incloure recomanacions espec fiques per a futurs treballs relacionats amb el problema tractat en aquesta mem ria.
\end{itemize}


\subsection{Ap ndixs}

Els ap ndixs contenen aquella informaci  que, tot i ser interessant que estigui en la mem ria del TFG, per un motiu o altre no  s apropiat que es trobi en el cos principal d'aquesta. Els motius poden ser molt diversos per  gaireb  tots ells estarien relacionats amb la continu tat del fil conductor de l'argumentaci  presentada en el cos principal de la mem ria. Per exemple, demostracions matem tiques molt extenses, taules completes de les caracter stiques dels models utilitzats en les simulacions, especificacions t cniques dels components, grans taules de dades o el codi font d'un algorisme, s n bons candidats per posar en un ap ndix.

\subsection{Refer ncies bibliogr fiques}

El coneixement cient fic, com qualsevol altre,  s acumulatiu i, per tant,  s normal que a mesura que anem escrivint la mem ria del TFG ho fem recolzant-nos en llibres, articles de revista, articles publicats en les actes d'un congr s, treballs in dits, \ldots d'altres autors. Aix   s completament ``legal'' (no podrem ser acusats de plagi) sempre que citem de forma adequada les fonts bibliogr fiques utilitzades. Aquestes refer ncies bibliogr fiques serviran, entre d'altres coses, per:
\begin{itemize}
   \item donar suport a les nostres reivindicacions o augmentar la credibilitat de les nostres argumentacions,
   \item referenciar els antecedents que ens han portat fins a la feina que presentem en aquest TFG,
   \item donar exemples de diferents punts de vista sobre un tema determinat,
   \item cridar l'atenci  sobre una posici  amb la que volem mostrar el nostre acord o desacord, o
   \item destacar una frase o un passatge especialment rellevant tot citant la font original.
\end{itemize}
