\chapter{La presentaci  del Treball Final de Grau}\label{presentaci }

\section{Principis b sics}
\emph{Qu   s el que fa que una presentaci  de \ac{TFG} sigui bona?} L'objectiu de qualsevol presentaci   s donar a con ixer un determinat missatge fent que l'audi ncia l'entengui i el recordi \cite{Blair91,Padgett08}. En concret, la presentaci  del \ac{TFG} ha de permetre a l'alumne sintetitzar el treball que ha estat realitzant durant varis mesos i donar-lo a con ixer als membres del tribunal per tal que aquests puguin determinar fins a quin punt s'han assolit els objectius plantejats.

Les qualitats m s rellevants que ha de tenir una presentaci  t cnica/cient fica per ser bona s n (\cite{Blair91,Padgett08,Arcy88,Rice04}):
\begin{itemize}\tightlist
  \item concisa i precisa: cal identificar les idees clau (objectius i assoliments del treball realitzat) i comunicar-les amb precisi  estil stica i t cnica.
  \item organitzada i estructurada: una organitzaci  acurada i coherent del material a presentar facilita enormement la seva comprensi  per part de l'audi ncia.
  \item adequada a l'audi ncia i capa  de captar i mantenir la seva atenci : s'ha d'\emph{enganxar} l'audi ncia per tal que entengui millor el que es vol expressar. Per aix  el missatge s'ha de caracteritzar per la seva:
      \begin{enumerate}\tightlist
        \item claredat
        \item simplicitat
        \item correcci 
        \item precisi 
      \end{enumerate}
  \item exhaustivament preparada: quan les idees es comuniquen d'una manera pobra no s'obt  cap benefici del nostre esfor , ni per part nostra ni per part de l'audi ncia.
\end{itemize}

L'objectiu ser  despertar prou inter s en el tribunal com per aconseguir captar veritablement la seva atenci  i donar-li a con ixer el problema a resoldre, la soluci  adoptada i els resultats obtinguts.

\section{Bones pr ctiques}
\emph{Quines s n les millors passes a fer per preparar una bona presentaci  de \ac{TFG}?}

\subsection{Les transpar ncies}

        A l'hora de preparar la presentaci  del \ac{TFG} podem caure en la temptaci  d'explicar el contingut complet del nostre treball, per  l'objectiu ha de ser deixar clars tan sols els punts m s importants. Si aquests no s n massa nombrosos, ser  possible explicar-los de manera clara, mentre que si pretenem mostrar massa idees tan sols aconseguirem confondre l'audi ncia. Els detalls sobre el treball desenvolupat es troben a la mem ria del \ac{TFG} i no s'han d'explicar a la presentaci . A l'hora de preparar les explicacions s'ha de tenir present que les nostres idees sempre ens semblen molt m s simples a nosaltres que a aquells que encara no les entenen. La clau d'una bona presentaci  es fonamenta en descriure de manera adequada:
        \begin{itemize}\tightlist
            \item el problema a resoldre: les necessitats del projecte, la seva motivaci , explicant la situaci  i l'entorn en el qual neix, el perqu   s necessari
            \item la soluci  adoptada i els resultats obtinguts
            \item les conclusions del treball i les possibles l nies de futur
        \end{itemize}

         s fonamental tenir presents les caracter stiques de l'audi ncia i proporcionar-li la informaci  que necessita per tal que pugui entendre la presentaci . Sempre  s bo comen ar amb una revisi  d'aquells conceptes b sics que, tot i ser probablement coneguts per l'auditori, facilitaran la comprensi  de la presentaci . Com a part de la preparaci  s'han d'avaluar no nom s els coneixements sin  tamb  els interessos, necessitats i valors de l'audi ncia (criteris de puntuaci  del \ac{TFG}).

        Un cop identificats els punts clau que volem deixar clars i les caracter stiques de l'audi ncia, es realitzar  un esb s del contingut de les transpar ncies. A partir d'aquest esb s es va perfilant l'estructura de la presentaci , que ha de seguir un fil argumental l gic i coherent. El contingut de les transpar ncies haur  de marcar en tot moment com es va avan ant per aquest fil, evitant que l'audi ncia es perdi i deixi d'estar atenta. Mantenint sempre aquest prop sit en ment es decideixen els continguts de les transpar ncies, que nom s inclouran aquelles explicacions necess ries per a poder assolir el nostre objectiu, que no  s altre que el de transmetre a l'auditori els punts que hem escollit com a idees clau. Aquells continguts que no siguin imprescindibles per a aconseguir aquesta finalitat no s'inclouran en les transpar ncies.

        Tot i que la normativa estableix una durada m xima de 45 minuts,  s recomanable que no excedeixi els 30 o 35 minuts, deixant uns 15 minuts pel torn de preguntes. S'han de redactar les transpar ncies tenint en compte que generalment  s recomanable dedicar aproximadament entre dos i tres minuts a cada transpar ncia \cite{Padgett08}. Cal tenir en compte que l'audi ncia sol perdre l'atenci  passats uns 10 o 15 minuts d'activitat similar \cite{Padgett08}. Aix  significa que el rellotge de l'atenci  s'ha de tornar a iniciar en el transcurs de l'exposici  oral, i per aconseguir-ho cal donar algun tipus de gir en la presentaci , la qual cosa es pot aconseguir, per exemple, amb algun tipus de demostraci  o de recapitulaci .

        En la redacci  del text de les transpar ncies es procurar  que tant els t tols com les frases siguin directes i curts. Els par grafs han de ser breus i l'estil utilitzat ha de ser impersonal i objectiu. A m s s'ha de procurar minimitzar el text present a les transpar ncies i s'evitar  sempre copiar par grafs complets de la mem ria a la presentaci .  s fonamental assegurar-se de la total correcci  ortogr fica i gramatical.

        Les taules, gr fics, diagrames i imatges que mostren el que volem expressar, s n eines de comunicaci  molt valuoses. Cal indicar clarament a l'audi ncia el que es mostra en aquests elements visuals, mai s'ha de donar per suposat que ja ho sap. D'aquesta manera qui at n les nostres explicacions es beneficia d'escoltar i de veure simult niament el nostre missatge. Si s'ha de fer alguna demostraci  de l'aplicaci  o producte final  s convenient gravar-ho en v deo per tal d'evitar possibles inestabilitats del software o problemes amb els servidors.

        Conv  provar les diapositives en el projector i el PC de la sala on es far  la defensa, ja que poden canviar colors, ... Tamb  cal confirmar que la mida de lletra  s adequada i que els gr fics es visualitzen de manera clara i que s n inte\l.ligibles. S'evitaran els colors estridents i els dissenys de diapositiva que restin claredat al contingut de les transpar ncies.

\subsection{L'exposici  i la defensa}

         s fonamental preparar-se b  la presentaci , explicitant el que s'ha de dir a cada transpar ncia. L'alumne ha de tenir molt clares les explicacions que pret n donar als membres del tribunal, ja que en cas que tingui aspectes confusos mai podr  transmetre'ls amb claredat.  s recomanable practicar-la repetides vegades tant tot sol com davant d'altres persones, per tal d'agafar confian a i flu desa, aix  com per ajustar-se al temps disponible. S'evitar  recitar de mem ria la presentaci , per  s  que  s bo memoritzar algunes paraules o frases clau per a cada transpar ncia. De totes maneres, per tal de superar el p nic esc nic, que  s especialment gran a l'inici, s  pot ser convenient aprendre's les dues o tres primeres transpar ncies.  s aconsellable fer un assaig general de la defensa en la mateixa sala i amb projector uns quants dies abans de la presentaci  definitiva, amb l'assist ncia del director del \ac{TFG}, ja que aquest podr  fer totes aquelles recomanacions i correccions que consideri oportunes i que seran de gran ajut per l'alumne.

        L'estil i el to de l'exposici  oral han d'afavorir que l'audi ncia mantingui la seva atenci . Per aix  el llenguatge utilitzat ha de ser apropiat no nom s a la disciplina sin  tamb  a l'audi ncia. S'utilitzar  un llenguatge t cnicament correcte, evitant utilitzar expressions co\l.loquials, repeticions i falques. S'ha de parlar amb autoritat i confian a, mostrant i demostrant el coneixement i domini del treball presentat. S'ha de tenir esment al to de veu i al ritme amb qu  es parla, utilitzant pauses, per exemple entre les distintes seccions, que afavoreixin la comprensi  de l'exposici . L'orador ha de fer un  s adequat del contacte visual, tant de manera individual com de cap al grup que l'est  escoltant. De la mateixa manera la posici  corporal i els gestos, aix  com l'aparen a, hauran de ser apropiats. No  s convenient moure's excessivament durant la presentaci , ni interposar-se entre la pantalla on es projecten les diapositives i els membres del tribunal. Un punter pot ajudar a centrar l'atenci  sobre algun punt espec fic d'una transpar ncia, per  no s'ha abusar d'aquest recurs.

        En general, qui m s sap del TGF  s el propi projectista, a part del seu director, de manera que el torn de preguntes dels membres del tribunal no ha de provocar cap temor en l'alumne.  s convenient anar a totes les defenses de \ac{TFG} possibles, i fixar-se en all  que l'alumne fa b  i malament, el tipus de preguntes que fa el tribunal, les respostes, \ldots\  s recomanable dur aigua a la presentaci .

\section{Estructura de la presentaci  del Treball Final de Grau}

L'estructura t pica de la presentaci  constar , de la mateixa manera que la mem ria, de quatre parts fonamentals: introducci , desenvolupament, resultats i conclusions. Tot i aix   s recomanable completar aquesta estructura b sica comen ant amb una portada seguida d'un  ndex.
\subsection{Portada} Inclour  el t tol del \ac{TFG} aix  com el nom dels autors, dels directors i dels tutors si s'escau, del departament, de la universitat, del t tol acad mic al qual s'opta i de la data de la presentaci . Una bona opci   s mantenir projectada des de l'inici aquesta primera transpar ncia, mentre els membres del tribunal i l'audi ncia entren a la sala.
\subsection{ ndex} Consistir  en una breu descripci  dels principals punts de la presentaci . Pret n preparar a l'audi ncia per tal que vagi identificant f cilment els punts importants a mesura que es van cobrint al llarg de la presentaci . Permet comunicar l'organitzaci  de la presentaci .  s bo que el gui  estigui sempre visualment present, per exemple a la cap alera de les transpar ncies.
\subsection{Desenvolupament} Explicar  el treball realitzat (problema a resoldre i soluci  adoptada), seguint un fil argumental l gic i coherent, i de manera que capti i mantingui l'atenci  i l'inter s de l'audi ncia.
\subsection{Resultats} Mostraran la validesa de la soluci  adoptada, generalment amb l'ajuda de taules i figures. Per assegurar la seva comprensi  per part de l'audi ncia  s imprescindible explicitar all  que es presenta en les taules i figures. A m s  s necessari assegurar-se de la seva correcta visualitzaci  amb el projector. No  s necessari mostrar tots els resultats presentats a la mem ria, sin  tan sols aquells que condueixin a la correcta comprensi  del treball realitzat i dels resultats obtinguts.
\subsection{Conclusions} Resumiran la presentaci  completa. Despr s d'haver explicat els principals punts en el cos de la presentaci , l'audi ncia pot haver-se perdut en els detalls, aix   s que cal reiterar la import ncia del problema que hav em plantejat i els principals aspectes de la soluci  proposada. Aix  mateix, tamb  es poden descriure breument les possibles l nies de treball futur.
